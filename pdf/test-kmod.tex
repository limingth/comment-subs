\documentclass[11pt,a4paper]{article}
\usepackage{fontspec}
\setmainfont{AR PL UKai CN} 
\usepackage{fontspec}
\usepackage{xunicode}
\usepackage{xltxtra}
\usepackage{indentfirst}
\XeTeXlinebreaklocale “zh”
\XeTeXlinebreakskip = 0pt plus 1pt minus 0.1pt

\usepackage{ifxetex,ifluatex}
\ifxetex
  \usepackage{fontspec,xltxtra,xunicode}
  \defaultfontfeatures{Mapping=tex-text,Scale=MatchLowercase}
\else
  \ifluatex
    \usepackage{fontspec}
    \defaultfontfeatures{Mapping=tex-text,Scale=MatchLowercase}
  \else
    \usepackage[utf8]{inputenc}
  \fi
\fi
\ifxetex
  \usepackage[setpagesize=false, % page size defined by xetex
              unicode=false, % unicode breaks when used with xetex
              xetex,
              colorlinks=true,
              linkcolor=blue]{hyperref}
\else
  \usepackage[unicode=true,
              colorlinks=true,
              linkcolor=blue]{hyperref}
\fi
\hypersetup{breaklinks=true, pdfborder={0 0 0}}
\setlength{\parindent}{0pt}
\setlength{\parskip}{6pt plus 2pt minus 1pt}
\setlength{\emergencystretch}{3em}  % prevent overfull lines
\setcounter{secnumdepth}{0}


\usepackage{framed,color}
\definecolor{shadecolor}{gray}{0.95}
\begin{document}

\section{kmod-11 测试}

\subsection{THU-12-1 下载编译源码包}

\subsubsection{下载源码包}

{\begin{shaded}\begin{verbatim}
$ wget https://www.kernel.org/pub/linux/utils/kernel/kmod/kmod-11.tar.gz
--2013-06-18 01:08:47--  https://www.kernel.org/pub/linux/utils/kernel/kmod/kmod-11.tar.gz
Resolving www.kernel.org (www.kernel.org)... 149.20.4.69, 198.145.20.140
Connecting to www.kernel.org (www.kernel.org)|149.20.4.69|:443... connected.
HTTP request sent, awaiting response... 200 OK
Length: 3458574 (3.3M) [application/x-gzip]
Saving to: `kmod-11.tar.gz'

100%[======================================>] 3,458,574    795K/s   in 5.3s    

2013-06-18 01:08:54 (641 KB/s) - `kmod-11.tar.gz' saved [3458574/3458574]
\end{verbatim}\end{shaded}}
查看下载文件大小 3458574字节

{\begin{shaded}\begin{verbatim}
$ ls -l
total 3380
-rw-rw-r-- 1 akaedu akaedu 3458574 Nov  8  2012 kmod-11.tar.gz
$ 
\end{verbatim}\end{shaded}}
\subsubsection{解压源码包}

{\begin{shaded}\begin{verbatim}
$ tar zxf kmod-11.tar.gz 
$ ls
kmod-11  kmod-11.tar.gz
\end{verbatim}\end{shaded}}
查看解压后文件数量 421个

{\begin{shaded}\begin{verbatim}
$ cd kmod-11
$ find * | wc  -l
421
$ 
\end{verbatim}\end{shaded}}
\subsubsection{编译项目源码}

{\begin{shaded}\begin{verbatim}
$ ./configure CFLAGS="-g -O2" --prefix=/usr --sysconfdir=/etc --libdir=/usr/lib
\end{verbatim}\end{shaded}}
查看当前目录下文件数量 确认已经生成 Makefile

{\begin{shaded}\begin{verbatim}
$ find * | wc  -l
467
$ ls -l Makefile
-rw-rw-r-- 1 akaedu akaedu 91795 Jun 18 01:15 Makefile
$ make
\end{verbatim}\end{shaded}}
查看tools目录下生成的可执行文件 \$ ls ./tools/ -l \textbar{} grep ``x''
lrwxrwxrwx 1 akaedu akaedu 10 Jun 18 01:17 depmod -\textgreater{}
kmod-nolib lrwxrwxrwx 1 akaedu akaedu 10 Jun 18 01:17 insmod
-\textgreater{} kmod-nolib -rwxrwxr-x 1 akaedu akaedu 8352 Jun 18 01:17
kmod -rwxrwxr-x 1 akaedu akaedu 468079 Jun 18 01:17 kmod-nolib
lrwxrwxrwx 1 akaedu akaedu 10 Jun 18 01:17 lsmod -\textgreater{}
kmod-nolib lrwxrwxrwx 1 akaedu akaedu 10 Jun 18 01:17 modinfo
-\textgreater{} kmod-nolib lrwxrwxrwx 1 akaedu akaedu 10 Jun 18 01:17
modprobe -\textgreater{} kmod-nolib lrwxrwxrwx 1 akaedu akaedu 10 Jun 18
01:17 rmmod -\textgreater{} kmod-nolib

安装可执行文件 \$ make install 权限不够,需要 sudo

{\begin{shaded}\begin{verbatim}
$ sudo make install
[sudo] password for akaedu: 
Making install in .
 /bin/mkdir -p '/usr/lib'
 /bin/bash ./libtool   --mode=install /usr/bin/install -c   libkmod/libkmod.la '/usr/lib'
libtool: install: /usr/bin/install -c libkmod/.libs/libkmod.so.2.2.1 /usr/lib/libkmod.so.2.2.1
libtool: install: (cd /usr/lib && { ln -s -f libkmod.so.2.2.1 libkmod.so.2 || { rm -f libkmod.so.2 && ln -s libkmod.so.2.2.1 libkmod.so.2; }; })
libtool: install: (cd /usr/lib && { ln -s -f libkmod.so.2.2.1 libkmod.so || { rm -f libkmod.so && ln -s libkmod.so.2.2.1 libkmod.so; }; })
libtool: install: /usr/bin/install -c libkmod/.libs/libkmod.lai /usr/lib/libkmod.la
libtool: finish: PATH="/usr/local/sbin:/usr/local/bin:/usr/sbin:/usr/bin:/sbin:/bin:/sbin" ldconfig -n /usr/lib
----------------------------------------------------------------------
Libraries have been installed in:
   /usr/lib

If you ever happen to want to link against installed libraries
in a given directory, LIBDIR, you must either use libtool, and
specify the full pathname of the library, or use the `-LLIBDIR'
flag during linking and do at least one of the following:
   - add LIBDIR to the `LD_LIBRARY_PATH' environment variable
     during execution
   - add LIBDIR to the `LD_RUN_PATH' environment variable
     during linking
   - use the `-Wl,-rpath -Wl,LIBDIR' linker flag
   - have your system administrator add LIBDIR to `/etc/ld.so.conf'

See any operating system documentation about shared libraries for
more information, such as the ld(1) and ld.so(8) manual pages.
----------------------------------------------------------------------
 /bin/mkdir -p '/usr/bin'
  /bin/bash ./libtool   --mode=install /usr/bin/install -c tools/kmod '/usr/bin'
libtool: install: /usr/bin/install -c tools/.libs/kmod /usr/bin/kmod
make --no-print-directory install-exec-hook
if test "/usr/lib" != "/usr/lib"; then \
        /bin/mkdir -p /usr/lib && \
        so_img_name=$(readlink /usr/lib/libkmod.so) && \
        so_img_rel_target_prefix=$(echo /usr/lib | sed 's,\(^/\|\)[^/][^/]*,..,g') && \
        ln -sf $so_img_rel_target_prefix/usr/lib/$so_img_name /usr/lib/libkmod.so && \
        mv /usr/lib/libkmod.so.* /usr/lib; \
    fi
 /bin/mkdir -p '/usr/include'
 /usr/bin/install -c -m 644 libkmod/libkmod.h '/usr/include'
 /bin/mkdir -p '/usr/lib/pkgconfig'
 /usr/bin/install -c -m 644 libkmod/libkmod.pc '/usr/lib/pkgconfig'
Making install in libkmod/docs
make[2]: Nothing to be done for `install-exec-am'.
make[2]: Nothing to be done for `install-data-am'.
Making install in man
make[2]: Nothing to be done for `install-exec-am'.
 /bin/mkdir -p '/usr/share/man/man5'
 /usr/bin/install -c -m 644 depmod.d.5 modprobe.d.5 modules.dep.5 modules.dep.bin.5 '/usr/share/man/man5'
 /bin/mkdir -p '/usr/share/man/man8'
 /usr/bin/install -c -m 644 depmod.8 insmod.8 lsmod.8 rmmod.8 modprobe.8 modinfo.8 '/usr/share/man/man8'
\end{verbatim}\end{shaded}}
\subsection{THU-12-2 测试 insmod 命令}

\subsubsection{编写测试用内核模块源码 hello.c}

{\begin{shaded}\begin{verbatim}
$ cat hello.c 

#include <linux/module.h>
#include <linux/kernel.h>

MODULE_AUTHOR("AKAEDU");
MODULE_DESCRIPTION("module example ");
MODULE_LICENSE("GPL");

int global = 100;

int __init 
akae_init (void)
{
    int local = 200;
    printk ("Hello, akaedu\n");

    printk(".text = %p\n", akae_init);
    printk(".data = %p\n", &global);
    printk(".stack = %p\n", &local);
    return 0;
}

void __exit
akae_exit (void)
{
    int local = 300;
    printk ("module exit\n");

    printk(".text = %p\n", akae_exit);
    printk(".data = %p\n", &global);
    printk(".stack = %p\n", &local);
    return ;
}

module_init(akae_init);
module_exit(akae_exit);
$ 
\end{verbatim}\end{shaded}}
\subsubsection{编写测试用内核模块的 Makefile 文件}

{\begin{shaded}\begin{verbatim}
$ cat Makefile 

obj-m := hello.o

KDIR := /usr/src/linux-headers-3.2.0-29-generic-pae/

all:
    make -C $(KDIR) SUBDIRS=$(PWD)  modules

clean:
    rm -rf *.o *.ko *.mod.* *.cmd 
    rm -rf .*

$ 
\end{verbatim}\end{shaded}}
\subsubsection{编译内核模块 hello.ko}

{\begin{shaded}\begin{verbatim}
$ cd hello-module/ 
$ make
make -C /usr/src/linux-headers-3.2.0-29-generic-pae/    SUBDIRS=/home/akaedu/Github/comment-subs/hello-module   modules
make[1]: Entering directory `/usr/src/linux-headers-3.2.0-29-generic-pae'
  CC [M]  /home/akaedu/Github/comment-subs/hello-module/hello.o
  Building modules, stage 2.
  MODPOST 1 modules
  CC      /home/akaedu/Github/comment-subs/hello-module/hello.mod.o
  LD [M]  /home/akaedu/Github/comment-subs/hello-module/hello.ko
make[1]: Leaving directory `/usr/src/linux-headers-3.2.0-29-generic-pae'
$ 
\end{verbatim}\end{shaded}}
\subsubsection{使用测试用工具 insmod 插入内核模块}

{\begin{shaded}\begin{verbatim}
$ sudo ./kmod-11/tools/insmod hello-module/hello.ko 
\end{verbatim}\end{shaded}}
\subsubsection{查看插入内核模块后的打印结果}

{\begin{shaded}\begin{verbatim}
$ lsmod | grep hello
hello                  12415  0 
$ dmesg | tail
[350775.859640] usb 2-2.1: USB disconnect, device number 14
[350777.611134] Bluetooth: hci0 urb c7304180 submission failed
[350778.217886] usb 2-2.1: new full-speed USB device number 15 using uhci_hcd
[352048.604051] usb 2-2.1: USB disconnect, device number 15
[352048.630829] Bluetooth: hci0 urb dd3d3000 submission failed
[352049.254135] usb 2-2.1: new full-speed USB device number 16 using uhci_hcd
[352111.505217] Hello, akaedu
[352111.505223] .text = e0844000
[352111.505225] .data = e0c03000
[352111.505227] .stack = df6e3f54
$ 
\end{verbatim}\end{shaded}}
\subsubsection{重复插入同样的内核模块系统会报错}

{\begin{shaded}\begin{verbatim}
$ sudo ./kmod-11/tools/insmod hello-module/hello.ko 
insmod: ERROR: could not insert module hello-module/hello.ko: File exists
$ lsmod | grep hello
hello                  12415  0 
\end{verbatim}\end{shaded}}
\subsection{THU-12-3 测试 rmmod 命令}

\subsubsection{使用测试用工具 rmmod 卸载内核模块}

{\begin{shaded}\begin{verbatim}
$ sudo ./kmod-11/tools/rmmod hello-module/hello.ko
$ (rmmod 命令的执行,运行在 hello 的后面加上 .ko 的后缀,这个和以前的命令有所不同)
\end{verbatim}\end{shaded}}
\subsubsection{查看卸载内核模块后的打印结果}

{\begin{shaded}\begin{verbatim}
$ lsmod | grep hello
$ (可以看到上面命令的执行结果没有任何输出信息)
$ dmesg | tail
[352048.630829] Bluetooth: hci0 urb dd3d3000 submission failed
[352049.254135] usb 2-2.1: new full-speed USB device number 16 using uhci_hcd
[352111.505217] Hello, akaedu
[352111.505223] .text = e0844000
[352111.505225] .data = e0c03000
[352111.505227] .stack = df6e3f54
[352365.795618] module exit
[352365.795624] .text = e0c01000
[352365.795626] .data = e0c03000
[352365.795628] .stack = dd197f40
$ 
\end{verbatim}\end{shaded}}
\subsection{THU-12-4 测试 lsmod 命令}

\subsubsection{lsmod 命令运行}

不加参数,直接运行 lsmod ,可以显示出当前在内核中的模块情况。

{\begin{shaded}\begin{verbatim}
$ ./kmod-11/tools/lsmod 
Module                  Size  Used by
nls_iso8859_1          12617  0 
nls_cp437              12751  0 
usb_storage            39646  0 
btrfs                 638208  0 
zlib_deflate           26622  1 btrfs
libcrc32c              12543  1 btrfs
ufs                    78131  0 
qnx4                   13309  0 
hfsplus                83507  0 
hfs                    49479  0 
minix                  31418  0 
ntfs                  100171  0 
vfat                   17308  0 
msdos                  17132  0 
fat                    55605  2 msdos,vfat
jfs                   175085  0 
xfs                   747494  0 
reiserfs              230896  0 
ext2                   67987  0 
usblp                  17885  0 
vmwgfx                102138  2 
ttm                    65344  1 vmwgfx
drm                   197692  3 ttm,vmwgfx
acpiphp                23535  0 
vmw_balloon            12700  0 
psmouse                72919  0 
serio_raw              13027  0 
btusb                  17912  2 
joydev                 17393  0 
rfcomm                 38139  0 
bnep                   17830  2 
bluetooth             158438  13 bnep,rfcomm,btusb
ppdev                  12849  0 
nfsd                  229850  13 
nfs                   307376  0 
lockd                  78804  2 nfs,nfsd
fscache                50642  1 nfs
i2c_piix4              13093  0 
auth_rpcgss            39597  2 nfs,nfsd
nfs_acl                12771  2 nfs,nfsd
sunrpc                205647  19 nfs_acl,auth_rpcgss,lockd,nfs,nfsd
parport_pc             32114  1 
shpchp                 32325  0 
mac_hid                13077  0 
snd_ens1371            24819  4 
gameport               15060  1 snd_ens1371
snd_rawmidi            25424  1 snd_ens1371
snd_seq_device         14172  1 snd_rawmidi
snd_ac97_codec        106082  1 snd_ens1371
ac97_bus               12642  1 snd_ac97_codec
snd_pcm                80845  3 snd_ac97_codec,snd_ens1371
snd_timer              28931  2 snd_pcm
snd                    62064  12 snd_timer,snd_pcm,snd_ac97_codec,snd_seq_device,snd_rawmidi,snd_ens1371
soundcore              14635  1 snd
snd_page_alloc         14108  1 snd_pcm
lp                     17455  0 
parport                40930  3 lp,parport_pc,ppdev
pcnet32                41110  0 
usbhid                 41906  0 
hid                    77367  1 usbhid
mptspi                 22474  2 
mptscsih               39530  1 mptspi
mptbase                96852  2 mptscsih,mptspi
floppy                 60310  0 
vmw_pvscsi             18334  0 
vmxnet3                44924  0 
$ 
\end{verbatim}\end{shaded}}
\subsubsection{lsmod 命令运行参数}

该命令不支持带参数,因此后面如果跟某个模块名称,只会显示 usage
,不会显示模块的信息。

{\begin{shaded}\begin{verbatim}
$ ./kmod-11/tools/lsmod ufs
Usage: ./kmod-11/tools/lsmod
$
\end{verbatim}\end{shaded}}
\subsection{THU-12-5 测试 modinfo 命令}

\subsubsection{modinfo 命令运行参数}

对于没有依赖关系的单个 .ko 内核模块,使用 modinfo
可以直接显示出模块的信息。

{\begin{shaded}\begin{verbatim}
$ ./kmod-11/tools/modinfo ./hello-module/hello.ko 
filename:       ./hello-module/hello.ko
license:        GPL
description:    module example 
author:         AKAEDU
srcversion:     C928237C5C93794C5E0EF9C
depends:        
vermagic:       3.2.0-29-generic-pae SMP mod_unload modversions 686 
$ 
\end{verbatim}\end{shaded}}
\subsubsection{modinfo 命令检查依赖关系}

对于有依赖关系的单个 .ko 内核模块,使用 modinfo
可以显示出模块的依赖关系信息depends,同时也可以显示出模块加载时的参数信息parm。这个参数信息是在编译内核模块的时候,源码中通过用
MODULE\_PARM\_DESC() 宏来指定的。

{\begin{shaded}\begin{verbatim}
$ modinfo /lib/modules/3.2.0-29-generic-pae/kernel/fs/nfs/nfs.ko 
filename:       /lib/modules/3.2.0-29-generic-pae/kernel/fs/nfs/nfs.ko
license:        GPL
author:         Olaf Kirch <okir@monad.swb.de>
srcversion:     BB0605CB0AF0BA47415CBEC
depends:        fscache,sunrpc,lockd,auth_rpcgss,nfs_acl
intree:         Y
vermagic:       3.2.0-29-generic-pae SMP mod_unload modversions 686 
parm:           callback_tcpport:portnr
parm:           cache_getent:Path to the client cache upcall program (string)
parm:           cache_getent_timeout:Timeout (in seconds) after which the cache upcall is assumed to have failed (ulong)
parm:           enable_ino64:bool
parm:           nfs4_disable_idmapping:Turn off NFSv4 idmapping when using 'sec=sys' (bool)
$ 
\end{verbatim}\end{shaded}}
\subsubsection{查看别名信息 alias}

对于可以使用别名的内核模块,也可以用它的别名 alias
来查看模块信息。别名是在 /lib/modules/3.2.0-29-generic-pae/modules.alias
描述的模块名称的简单形式。

{\begin{shaded}\begin{verbatim}
$ head /lib/modules/3.2.0-29-generic-pae/modules.alias
# Aliases extracted from modules themselves.
alias pci:v00008086d00003422sv*sd*bc*sc*i* mce_xeon75xx
alias char-major-10-134 apm
alias devname:cpu/microcode microcode
alias char-major-10-184 microcode
alias aes-asm aes_i586
alias aes aes_i586
alias twofish-asm twofish_i586
alias twofish twofish_i586
alias salsa20-asm salsa20_i586
\end{verbatim}\end{shaded}}
但是在这个文件中,别名为 aes 的模块,还有很多个,通过 grep ``alias aes''
可以看出一共有3个别名都是 aes 的模块,分别是 aes\_i586, aesni\_intel,
padlock\_aes。

{\begin{shaded}\begin{verbatim}
$ cat /lib/modules/3.2.0-29-generic-pae/modules.alias | grep "alias aes"
alias aes-asm aes_i586
alias aes aes_i586
alias aes aesni_intel
alias aes padlock_aes
\end{verbatim}\end{shaded}}
\subsubsection{使用 modinfo 查看各个依赖模块信息}

通过 modinfo 来查看 aes
这个别名所对应的模块信息,可以看到这3个模块所对应的模块文件
crypto/aes-i586.ko,aesni-intel.ko,padlock-aes.ko 的详细信息。

{\begin{shaded}\begin{verbatim}
$ ./kmod-11/tools/modinfo aes 
filename:       /lib/modules/3.2.0-29-generic-pae/kernel/arch/x86/crypto/aes-i586.ko
alias:          aes-asm
alias:          aes
license:        GPL
description:    Rijndael (AES) Cipher Algorithm, asm optimized
srcversion:     24373C7FF739526E8AAF1B0
depends:        
intree:         Y
vermagic:       3.2.0-29-generic-pae SMP mod_unload modversions 686 

filename:       /lib/modules/3.2.0-29-generic-pae/kernel/arch/x86/crypto/aesni-intel.ko
alias:          aes
license:        GPL
description:    Rijndael (AES) Cipher Algorithm, Intel AES-NI instructions optimized
srcversion:     E0B859CB1FF480D0B70F6F2
depends:        cryptd,aes-i586
intree:         Y
vermagic:       3.2.0-29-generic-pae SMP mod_unload modversions 686 

filename:       /lib/modules/3.2.0-29-generic-pae/kernel/drivers/crypto/padlock-aes.ko
alias:          aes
author:         Michal Ludvig
license:        GPL
description:    VIA PadLock AES algorithm support
srcversion:     6842B20FF8E68314ED45103
depends:        
intree:         Y
vermagic:       3.2.0-29-generic-pae SMP mod_unload modversions 686 
$ 
\end{verbatim}\end{shaded}}
\subsection{THU-12-6 测试 depmod 命令}

\subsubsection{depmod 命令运行时调试图}

{\begin{shaded}\begin{verbatim}
$ ./kmod-11/tools/depmod | wc -l
3529

$ vi ./kmod-11/tools/depmod.c
修改源码文件,在 output_deps 函数中间插入打印函数,打印输出到标准输出 stdout。

1790 static int output_deps(struct depmod *depmod, FILE *out)
1791 {
1792         size_t i;
1793 
1794         fprintf(stdout, "total count %d", depmod->modules.count);
1795 ...
1798                 const char *p = mod_get_compressed_path(mod);
1799                 size_t j, n_deps;
1800 
1801                 if (mod->dep_loop) {
1802                         DBG("Ignored %s due dependency loops\n", p);
1803                         continue;
1804                 }
1805 
1806                 fprintf(out, "%s:", p);
1807                 fprintf(stdout, "%s:", p);

$ make -C kmod-11
make[1]: Entering directory `/home/akaedu/Github/comment-subs/kmod-11'
make --no-print-directory all-recursive
Making all in .
  CC       tools/depmod.o
  CCLD     tools/kmod
  CCLD     tools/kmod-nolib
Making all in libkmod/docs
make[3]: Nothing to be done for `all'.
Making all in man
make[3]: Nothing to be done for `all'.
make[1]: Leaving directory `/home/akaedu/Github/comment-subs/kmod-11'

$ sudo ./kmod-11/tools/depmod | head
total count 3529
kernel/arch/x86/kernel/cpu/mcheck/mce-xeon75xx.ko:
kernel/arch/x86/kernel/cpu/mcheck/mce-inject.ko:
kernel/arch/x86/kernel/msr.ko:
kernel/arch/x86/kernel/cpuid.ko:
kernel/arch/x86/kernel/apm.ko:
kernel/arch/x86/kernel/microcode.ko:
kernel/arch/x86/crypto/aes-i586.ko:
kernel/arch/x86/crypto/twofish-i586.ko: kernel/crypto/twofish_common.ko
kernel/arch/x86/crypto/salsa20-i586.ko:
kernel/arch/x86/crypto/aesni-intel.ko: kernel/arch/x86/crypto/aes-i586.ko kernel/crypto/cryptd.ko
$ 
\end{verbatim}\end{shaded}}
\subsection{THU-12-7 测试 depmod 命令}

\subsubsection{modprobe 命令运行时调试图}

{\begin{shaded}\begin{verbatim}
$ sudo ./kmod-11/tools/modprobe -r nfs
name = nfs

line = kernel/fs/nfs/nfs.ko: kernel/fs/nfs_common/nfs_acl.ko kernel/net/sunrpc/auth_gss/auth_rpcgss.ko kernel/fs/fscache/fscache.ko kernel/fs/lockd/lockd.ko kernel/net/sunrpc/sunrpc.ko
---------------------

p = kernel/fs/nfs_common/nfs_acl.ko
---------------------

p = kernel/net/sunrpc/auth_gss/auth_rpcgss.ko
---------------------

p = kernel/fs/fscache/fscache.ko
---------------------

p = kernel/fs/lockd/lockd.ko
---------------------

p = kernel/net/sunrpc/sunrpc.ko
---------------------
$ 

$ sudo ./kmod-11/tools/modprobe nfs
name = nfs

line = kernel/fs/nfs/nfs.ko: kernel/fs/nfs_common/nfs_acl.ko kernel/net/sunrpc/auth_gss/auth_rpcgss.ko kernel/fs/fscache/fscache.ko kernel/fs/lockd/lockd.ko kernel/net/sunrpc/sunrpc.ko
---------------------

p = kernel/fs/nfs_common/nfs_acl.ko
---------------------

p = kernel/net/sunrpc/auth_gss/auth_rpcgss.ko
---------------------

p = kernel/fs/fscache/fscache.ko
---------------------

p = kernel/fs/lockd/lockd.ko
---------------------

p = kernel/net/sunrpc/sunrpc.ko
---------------------
$ 
\end{verbatim}\end{shaded}}
\subsection{THU-12-8 编译生成 testsuite 命令集}

\subsubsection{}

{\begin{shaded}\begin{verbatim}
$ make check
Making check in .
  GEN      rootfs
make --no-print-directory testsuite/uname.la testsuite/path.la testsuite/init_module.la testsuite/delete_module.la testsuite/libtestsuite.la testsuite/test-init testsuite/test-testsuite testsuite/test-loaded testsuite/test-modinfo testsuite/test-alias testsuite/test-new-module testsuite/test-modprobe testsuite/test-blacklist testsuite/test-dependencies testsuite/test-depmod
  CC       testsuite/uname.lo
  CCLD     testsuite/uname.la
  CC       testsuite/path.lo
  CCLD     testsuite/path.la
  CC       testsuite/init_module.lo
  CC       testsuite/mkdir.lo
  CCLD     testsuite/init_module.la
  CC       testsuite/delete_module.lo
  CCLD     testsuite/delete_module.la
  CC       testsuite/testsuite_libtestsuite_la-testsuite.lo
  CCLD     testsuite/libtestsuite.la
  CC       testsuite/testsuite_test_init-test-init.o
  CCLD     testsuite/test-init
  CC       testsuite/testsuite_test_testsuite-test-testsuite.o
  CCLD     testsuite/test-testsuite
  CC       testsuite/testsuite_test_loaded-test-loaded.o
  CCLD     testsuite/test-loaded
  CC       testsuite/testsuite_test_modinfo-test-modinfo.o
  CCLD     testsuite/test-modinfo
  CC       testsuite/testsuite_test_alias-test-alias.o
  CCLD     testsuite/test-alias
  CC       testsuite/testsuite_test_new_module-test-new-module.o
  CCLD     testsuite/test-new-module
  CC       testsuite/testsuite_test_modprobe-test-modprobe.o
  CCLD     testsuite/test-modprobe
  CC       testsuite/testsuite_test_blacklist-test-blacklist.o
  CCLD     testsuite/test-blacklist
  CC       testsuite/testsuite_test_dependencies-test-dependencies.o
  CCLD     testsuite/test-dependencies
  CC       testsuite/testsuite_test_depmod-test-depmod.o
  CCLD     testsuite/test-depmod
make --no-print-directory check-TESTS
TESTSUITE: running test_initlib, in forked context
TESTSUITE: 'test_initlib' [20196] exited with return code 0
TESTSUITE: PASSED: test_initlib
TESTSUITE: running test_insert, in forked context
TESTSUITE: 'test_insert' [20197] exited with return code 0
TESTSUITE: PASSED: test_insert
TESTSUITE: running test_remove, in forked context
TESTSUITE: 'test_remove' [20198] exited with return code 0
TESTSUITE: PASSED: test_remove
PASS: testsuite/test-init
TESTSUITE: running testsuite_uname, in forked context
TRAP uname(): missing export TESTSUITE_UNAME_R?
TESTSUITE: 'testsuite_uname' [20202] exited with return code 0
TESTSUITE: PASSED: testsuite_uname
TESTSUITE: running testsuite_rootfs_fopen, in forked context
TESTSUITE: 'testsuite_rootfs_fopen' [20203] exited with return code 0
TESTSUITE: PASSED: testsuite_rootfs_fopen
TESTSUITE: running testsuite_rootfs_open, in forked context
TESTSUITE: 'testsuite_rootfs_open' [20204] exited with return code 0
TESTSUITE: PASSED: testsuite_rootfs_open
TESTSUITE: running testsuite_rootfs_stat_access, in forked context
TESTSUITE: 'testsuite_rootfs_stat_access' [20205] exited with return code 0
TESTSUITE: PASSED: testsuite_rootfs_stat_access
TESTSUITE: running testsuite_rootfs_opendir, in forked context
TESTSUITE: 'testsuite_rootfs_opendir' [20206] exited with return code 0
TESTSUITE: PASSED: testsuite_rootfs_opendir
PASS: testsuite/test-testsuite
TESTSUITE: running loaded_1, in forked context
TESTSUITE: 'loaded_1' [20210] exited with return code 0
TESTSUITE: PASSED: loaded_1
PASS: testsuite/test-loaded
TESTSUITE: running modinfo_jonsmodules, in forked context
TESTSUITE: 'modinfo_jonsmodules' [20214] exited with return code 0
TESTSUITE: PASSED: modinfo_jonsmodules
PASS: testsuite/test-modinfo
TESTSUITE: running alias_1, in forked context
TESTSUITE: 'alias_1' [20218] exited with return code 0
TESTSUITE: PASSED: alias_1
PASS: testsuite/test-alias
TESTSUITE: running from_name, in forked context
TESTSUITE: 'from_name' [20222] exited with return code 0
TESTSUITE: PASSED: from_name
TESTSUITE: running from_alias, in forked context
TESTSUITE: 'from_alias' [20223] exited with return code 0
TESTSUITE: PASSED: from_alias
PASS: testsuite/test-new-module
TESTSUITE: running modprobe_show_depends, in forked context
TESTSUITE: 'modprobe_show_depends' [20227] exited with return code 0
TESTSUITE: PASSED: modprobe_show_depends
TESTSUITE: running modprobe_show_depends2, in forked context
TESTSUITE: 'modprobe_show_depends2' [20228] exited with return code 0
TESTSUITE: PASSED: modprobe_show_depends2
TESTSUITE: running modprobe_builtin, in forked context
TESTSUITE: 'modprobe_builtin' [20229] exited with return code 0
TESTSUITE: PASSED: modprobe_builtin
TESTSUITE: running modprobe_softdep_loop, in forked context
TESTSUITE: 'modprobe_softdep_loop' [20230] exited with return code 0
TESTSUITE: PASSED: modprobe_softdep_loop
TESTSUITE: running modprobe_install_cmd_loop, in forked context
TESTSUITE: ERR: Test 'modprobe_install_cmd_loop' timed out, killing 20231
TESTSUITE: ERR: 'modprobe_install_cmd_loop' [20231] terminated by signal 9 (Killed)
FAIL: testsuite/test-modprobe
TESTSUITE: running blacklist_1, in forked context
TESTSUITE: 'blacklist_1' [20283] exited with return code 0
TESTSUITE: PASSED: blacklist_1
PASS: testsuite/test-blacklist
TESTSUITE: running test_dependencies, in forked context
TRAP uname(): missing export TESTSUITE_UNAME_R?
TESTSUITE: 'test_dependencies' [20287] exited with return code 0
TESTSUITE: PASSED: test_dependencies
PASS: testsuite/test-dependencies
TESTSUITE: running depmod_modules_order_for_compressed, in forked context
TESTSUITE: 'depmod_modules_order_for_compressed' [20291] exited with return code 0
TESTSUITE: ERR: sizes do not match /home/akaedu/Github/test-kmod-11/kmod-11/testsuite/rootfs/test-depmod/modules-order-compressed/lib/modules/3.5.4-1-ARCH/correct-modules.alias /home/akaedu/Github/test-kmod-11/kmod-11/testsuite/rootfs/test-depmod/modules-order-compressed/lib/modules/3.5.4-1-ARCH/modules.alias
TESTSUITE: ERR: FAILED: exit ok but outputs do not match: depmod_modules_order_for_compressed
FAIL: testsuite/test-depmod
==============================================
2 of 10 tests failed
Please report to linux-modules@vger.kernel.org
==============================================
make[2]: *** [check-TESTS] Error 1
make[1]: *** [check-am] Error 2
make: *** [check-recursive] Error 1
$ Killed

$ 
\end{verbatim}\end{shaded}}
\subsubsection{}

\end{document}
